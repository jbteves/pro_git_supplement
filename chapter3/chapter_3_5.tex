\subsubsection{Questions for Understanding}
\begin{enumerate}
	\item What flag do you use to override the default name \verb`origin` when running
	   \verb`git clone`?
	\item Why might there be several remotes for a project?
	\item State the difference between the following:
	   \begin{verbatim}
		   git checkout -b branch remote/branch
	   \end{verbatim}
	   and
	   \begin{verbatim}
	   git checkout --track remote/branch
	   \end{verbatim}
	   and then explain why you might choose one over the other.
	\item According to the text, it is preferable to use \verb`fetch` and
	\verb`merge`
	   rather than \verb`pull` by itself.
	   Why might it be confusing to use \verb`pull`?
\end{enumerate}

\subsubsection{Exercises}
Do exercises in order in the recipes repository.
\begin{enumerate}
\item Create a remote repository and add it as a remote.
   Call it \verb`test`.
   \begin{enumerate}
	   \item Create a branch, \verb`eggs`, and put a recipe for scrambled eggs on it.
		  Push it to \verb`test`.
	   \item Checkout to \verb`main` and create a new branch, \verb`fried_eggs`.
		  Push it to \verb`test`.
	   \item Delete your local \verb`eggs` and \verb`fried_eggs` branches, then fetch them
		  from \verb`test`.
		  Note: the ability to do this is one of the strengths of \verb`git`.
	   \item Merge \verb`eggs` and \verb`fried_eggs` into \verb`main`, and
	   push \verb`main` to the
		  \verb`test`.
	   \item Delete \verb`eggs` and \verb`fried_eggs` both locally and on the remote.
    \end{enumerate}
\end{enumerate}
